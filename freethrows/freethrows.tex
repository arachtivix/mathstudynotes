\documentclass{article}
\usepackage{amsmath}
\usepackage{amsfonts}
\usepackage{hyperref}

\title{Freethrow Percentage Problem}
\author{Daniel Werner}

\begin{document}

\maketitle

\section*{Problem Statement}

\

A basketball player has a freethrow percentage less than 75\%
early in the season and greater than 75\% later in the season.
Was there necessarily a point at which their freethrow
percentage was exactly 75\% at some point in the season?

\par

This problem was asked in a 
\href{https://www.youtube.com/watch?v=vqpWwgpDs58}{youtube video}.

\section*{Analysis}

The equation for freethrow percentage is fairly self evident
but it may be worth making more explicit.  The input for
the function is a sequence of zeros and ones, a one
representing making a freethrow and a one representing
missing.  Let's represent the sequence as T:

\begin{equation*}
    T = \{T_1, T_2, ..., T_n\}
\end{equation*}

The freethrow computation would be:

\begin{equation}
    F(n) = \frac{\sum_{k=0}^{n} T_k}{n}
\end{equation}

So the question amounts to: does there exist a sequence T
such that $F(n) > .75$ and $F(n - 1) < .75$?

\par

The first thing worth noting about this problem is that the
order of the throws before the nth does not matter.  In other
words, the sequence is actually irrelevant. Furthermore
the content of F(n) must be $\frac{1 + \sum_{k=0}^{n-1} T_k}{n}$
as per the framing of the question.  We can reduce
the question to whether there exists an a and b such
that all of the following are true:

\begin{align*}
    a, b \in \mathbb{N} \\
    \frac{a}{b} < \frac{3}{4} \\
    \frac{a + 1}{b + 1} > \frac{3}{4}
\end{align*}

It would follow that

\begin{align*}
    a < \frac{3}{4} \cdot b
\end{align*}

and

\begin{align*}
    a + 1 > \frac{3}{4} \cdot b + \frac{3}{4} \\
    a > \frac{3}{4} \cdot b - \frac{1}{4}
\end{align*}

which combined requires that

\begin{align*}
    \frac{3}{4} \cdot b - \frac{1}{4} < &a < \frac{3}{4} * b \\
    3b - 1 < &4a < 3b
\end{align*}

This equation cannot be satisfied because $3b$ is exactly
one greater than $3b - 1$.  There can be no natural number
between two other natural numbers that differ by one.
The necessary conclusion is that there must always be a
point in time under the given conditions at which the player
has a freethrow percentage of exactly 75.

\par

This of course does nothing to explain why this is the case
or whether there is some more general conclusion to draw.

\end{document}