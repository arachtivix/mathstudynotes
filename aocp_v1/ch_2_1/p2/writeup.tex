\documentclass{article}
\usepackage{amsmath}

\title{The Art Of Computer Programming -- Chapter 2.1, problem 2}
\author{Daniel Werner}

\begin{document}

\maketitle

\section*{
    Problem Statement
}

\begin{quote}
  The text points out that in many cases CONTENTS(LOC(V)) = V. Under what conditions do we have LOC(CONTENTS(V)) = V?
\end{quote}

\section*{
    Analysis
}

The first situation arises when a memory cell contains a value equal to its location.  The second situation gives LOC a paremeter that is a value, which makes no sense, as LOC can only operate on a link variable.


\end{document}
