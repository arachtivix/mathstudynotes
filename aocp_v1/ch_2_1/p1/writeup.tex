\documentclass{article}
\usepackage{amsmath}

\title{The Art of Computer Programming, Chapter 2.1, problem 1}
\author{Daniel Werner}

\begin{document}

\maketitle

\section*{
    Problem Statement
}

What is the value of a) SUIT(NEXT(TOP)) and b) NEXT(NEXT(NEXT(TOP))) in situation 3?  In the book, situation 3
contains three memory locations with cards designated by numerical values in MIX memory cells.

\section*{
    Answer
}

The cards structure represents a simple linked list structure with NEXT representing the traversal of one of The
links from one node/card to another.  The NEXT from the TOP is the second card, so NEXT(TOP) is the second card. SUIT()
of that card is 4, or diamonds.  For b), three iterations of next from top will be past the end of the list by one, so
one can call this $\Lambda$ as they do in the book.

\par

I have implemented a test case that loads the cards from scenario 3 into MIX memory and then runs a MIX program that
stores the results of scenario a in the x register.

\end{document}
