\documentclass{article}
\usepackage{amsmath}

\title{The Art of Computer Programming -- Chapter 2.2.1 Problem 4}
\author{Daniel Werner}

\begin{document}

\maketitle

\section*{
    Problem Statement
}

Find a simple formula for the number of permutations on n elements that can be obtained with a stack like that in exercise 2.

\section*{
    Analysis
}

Some information gathered from the previous problems:
\begin{enumerate}
  \item For a given input size $n$, there will be $2n$ characters in its 'S' and 'X' representation.
  \item Valid 'S' and 'X' representations can be identified by applying a simple counting rule that ensures there are the same number of S and X characters and that the running total of X never exceeds the running total of S.
\end{enumerate}

With this information, I will attempt to construct a recurrence that accurately models the 'S' and 'X' notation rules, then attempt to reduce that recurrence to a closed form.

\subsection*{Approach 1: special cases (dead end so far)}

One way I have considered to start on this problem is to consider special or simplifying cases first.  If the stack is limited in size, it could make the analysis easier by seeing what happens when its usage is allowed to increase.  I'll treat this like another parameter to our function, so whereas I am ultimately looking for $a_n$ as a function of $n$, for now I will look at $a_{m,n}$ where $m$ is the maximum stack size and $n$ is still the input size.  It is easy to see that:

\begin{equation*}
  a_{0,n} = 1
\end{equation*}

for any value of $n$.  This is because $a_{0,n}$ just means we never use the stack -- the output can only accept the same order as the input, which is exactly one permutation.  The question of $a_{1,n}$ is more interesting.  This function is like saying you can take any single input and move it as far back as you like:

\begin{equation*}
  \{ p_1,p_2,...p_{n-1},p_{n} \}
  \rightarrow
  \{ p_2,p_3,...p_{n-1},p_{n},p_{1} \}
\end{equation*}

This is not a complete description though, as when the new position for $a_1$ is less than $n$, there is the opportunity to move another element back as well:

\begin{equation*}
  \{ p_1,p_2,...p_{n-1},p_{n} \}
  \rightarrow
  \{ p_2,p_3,...,p_k,p_1,p_{k+1}...p_{n-1},p_{n} \}
\end{equation*}

The remaining tail sequence $\{p_{k+1},...,p_n\}$ can be treated the same recrusively as the larger sequence, this time with $n - k$ elements. So for $a_{1,n}$, you can look at it as having $n$ choices of where to put the first character in the sequence (including keeping it where it is) and walking a tree of possibilities from each choice.  This self similarity allows me to express all these choices as a sum:

\begin{equation}
  a_{1,n} = 1 + \sum_{k=1}^{n-1} a_{1,k}
\end{equation}

This sum nicely adheres to the clearly demonstrable base case of $a_{1,0} = 1$, as mentioned before since the number of permutations available for a zero length input is 1, regardless of maximum stack size.  While this is not even a decent form for the solution to $a_{1,n}$, it would be nice to check some values to make sure this is right.  Some modification on the script I wrote for problem 2 should help to find small values of this problem with the stack size restricted to 1.  Both the simulation and the sum seem to agree that:

\begin{equation*}
  a_{1,n} = 2^{n -1}
\end{equation*}

\par

This is interesting, but I do not see the path forward yet.

\subsection*{Approach 2: reasoning with the 'SX' form}

If I knew the insertion patterns for all valid permutations on $n - 1$, the only valid ones involving the nth element would be after $n - 1$ occurrences of S.  If the last S occurs in position $p_k$, then there would be $n - p + 1$ places for the additional S somewhere in the sequence and then another count of spaces available for the additional X that would have to occur (at any time) after that. Here are the valid patterns for 2-element inputs:

\begin{align*}
  SXSX = \{1~2\} \\
  SSXX = \{2~1\}
\end{align*}

Adding another input element, tracing back to which pattern can be said to be the parent, using lowercase for the parent and upper for the child:

\begin{align*}
  SXSX \rightarrow \{ SXSXsx, SXSsxX \} &= \{\{1~2~3\},\{1~3~2\}\} \\
  SSXX \rightarrow \{ SSsxXX,SSXsxX,SSXXsx \} &= \{\{3~2~1\},\{2~3~1\},\{2~1~3\}\}
\end{align*}

This seems quite nice, but not perfectly useful yet.  Since there are no remaining S's, this means only the location of the inserted S is meaningful, as the Xs do the same thing whether they are from the new pattern or the old.  The reason this does not seem immediately useful though is that there is still the question of how to count trailing Xs on the patterns from the permutations on $n - 1$ elements.

\par

I can see that this turns on the number of trailing Xs and it should be helpful to think of the patterns provided by adding an 'sx' to a pattern with k trailing X's will result in a single new pattern with k + 1 trailing X's, a single with k traling X's, and so forth down to a pattern with a single trailing x from the insertion itself.  SXSX yields two new patterns and SSXX yeilds three because the latter has one more trailing.

\par

For a given pattern ending in k X's, the number of resulting children is:

\begin{align*}
  direct~child~count=trailing~X~count + 1
\end{align*}

Each child will have a certain trailing X count and there will be one of each.  It turns out to be a little helpful to just ignore the characters preceding the last Xs and tally the counts of those with the same numer of last Xs.  Take the transition from $a_2$ to $a_3$:

\begin{align*}
  a_2 = \{SXSX,SSXX\} \rightarrow a_3 =
  \left\{
    \begin{aligned}
      & SXSXsx, SXSsxX, SSsxXX, \\
      & SSXsxX, SSXXsx
    \end{aligned}
  \right\}
\end{align*}

The same in abbreviated, grouped form:

\begin{align*}
  a_2 = \{1 * ...SX,1 * ...SXX\} \rightarrow a_3 =
  \left\{
    \begin{aligned}
      & 2 * ...SX, 2 * ...SXX, \\
      & 1 * ...SXXX
    \end{aligned}
  \right\}
\end{align*}

Because any ...SXXX could be turned into a ...SXXXX or a ...stay a SXXX, or get demoted to a SXX or a SX,

\begin{align*}
  \{1 * ...SXXX\} \rightarrow
  \left\{
    \begin{aligned}
      & 1 * ...SX, 1 * ...SXX, 1 * ...SXXX, \\
      & 1 * ...SXXXX
    \end{aligned}
  \right\}
\end{align*}

But this is not a scenario that happens by itself in the process.  Instead, we get the combined, where:



\begin{align*}
  a_3 = 
  \left\{
    \begin{aligned}
      & 2 * ...SX, 2 * ...SXX, \\
      & 1 * ...SXXX
    \end{aligned}
  \right\} \rightarrow a_4 =
  \left\{
    \begin{aligned}
      & 5 * ...SX, 5 * ...SXX, 3 * ...SXXX, \\
      & 1 * ...SXXXX
    \end{aligned}
  \right\}
\end{align*}

\par

A table will help to start the efforts to make this process possibly easier to track:

\begin{table}[h!]
\centering
 \begin{tabular}{|c | c | c | c | c | c | c|} 
 \hline
 $n$ & $a_n$ & ...SX & ...SXX & ...SXXX & ...SXXXX & ...SXXXXX \\ [0.5ex] 
 \hline
 1 & 1 & 1 &  &  &  & \\ 
 2 & 2 & 1 & 1 &  &  & \\
 3 & 5 & 2 & 2 & 1 &  & \\
 4 & 14 & 5 & 5 & 3 & 1 & \\
 5 & 42 & 14 & 14 & 9 & 4 & 1 \\
 \hline
 \end{tabular}
\end{table}

This table helps to regularize the problem, and I could write code to implement it, but the challenge to find a mathematical expression now faces the obstacle of how to deal with the terms in the table always expanding to the right.  It is almost like a set of child sequences within the parent sequence.  I will call this inner sequence $b_{n,m}$ where $n$ is the same as in $a_n$ but $m$ is now another parameter with valid values from 1 to $n$ inclusive.  For purposes of clarity, here's the table with the sub-sequence annotated:

\begin{table}[h!]
  \centering
   \begin{tabular}{|c | c | c | c | c | c | c|} 
   \hline
   $n$ & $a_n$ & ...SX & ...SXX & ...SXXX \\ [0.5ex] 
   \hline
   1 & 1 & $b_{1,1}$ &  & \\ 
   2 & 2 & $b_{2,1}$ & $b_{2,2}$ & \\
   3 & 5 & $b_{3,1}$ & $b_{3,2}$ & $b_{3,3}$ \\
   ... & ... & ... & ... & ... \\
   \hline
   \end{tabular}
\end{table}

\par

It seems pretty clear that

\begin{equation*}
  b_{n,1} = b_{n,2} = a_{n-1}
\end{equation*}

because the 'SX' count will be the same as the sum of all the previous generation of the parent series.  This is because any SX expression will have one child expression with one trailing X.  The same goes for two trailing Xs.

\par

For $m > 2$ the situation gets trickier.  Because any trailing X count can be arbitrarily reduced, any parent count greater than $m$ will contribute to the count on the next step, as will equal or minus 1.  In other words for $m > 2$:

\begin{equation*}
  b_{n,m} = \sum_{k=m-1}^{n - 1} b_{n-1,k}
\end{equation*}

The table then also helps us to remember that $a_n$ is the sum of all the $b_{n,m}$:

\begin{align*}
  a_{n} &= \sum_{k=1}^{n} b_{n,k} \\
  &= b_{1,n-1} + b_{2,n-1} + \sum_{k=3}^{n} b_{n,k} \\
  &= 2 a_{n-1} + \sum_{k=3}^{n} b_{n,k}
\end{align*}

So to consolidate the notes on this recurrence so far:

\begin{align*}
  a_0 = 1, a_1 = 1 \hspace{1cm}
  &a_n = 2 a_{n-1} + \sum_{k=3}^{n} b_{n,k} \\
  b_{n,1} = b_{n,2} = a_{n-1} \hspace{1cm}
  &b_{n,m} = \sum_{k=m-1}^{n-1} b_{n-1,k}
\end{align*}

The $b_n$ series does not really need a base case, as its definition refers back to $a_{n-1}$ and obtaining two starting values for $a_n$ is easy.  The sum in the definition of $b_{n,m}$ is zero for $n <= 2$, so it fits together nicely it seems.  All that this has accomplished though so far is to mathematically describe the problem.  The solution will depend on simplifying this set of expressions to a simpler form, but first, let's validate the expressions on a few iterations so that we do not attempt to solve the wrong recurrence.

\begin{equation*}
  a_2 = 2 * a_1 + \sum_{k=3}^{2} b_{2,k} = 2 + 0 = 2
\end{equation*}

So far so good.

\begin{align*}
  &b_{3,3} = \sum_{k=2}^{2} b_{2,k} = b_{2,2} = a_1 = 1 \\
  &a_3 = 2 * a_2 + \sum_{k=3}^{3} b_{3,k} = 2a_2 + b_{3,3} = 2 * 2 + 1 = 5
\end{align*}

Also good.

\begin{align*}
  &b_{4,4} = \sum_{k=3}^{3} b_{3,k} = b_{3,3} = 1 \\
  &b_{4,3} = \sum_{k=2}^{3} b_{3,k} = b_{3,2} + b_{3,3} = a_2 + 1 = 2 + 1 = 3 \\
  &a_4 = 2 a_{3} + \sum_{k=3}^{4} b_{4,k} = 2 * 5 + b_{4,3} + b_{4,4} = 10 + 3 + 1 = 14
\end{align*}

One observation I can easily make about this recurrence is that its closed form will have an exponential term.  The formula for $a_n$ plainly is a multiple of $a_{n-1}$ with some more added on after.  The sums in this recurrence are really cramping my style though.

One of the possibly questionable decisions I made above could be having attempted to define $b$ in terms of two parameters.  In any case, I've decided to view the book's solution, as this problem, although interesting so far, has proven less than amenable to the tools I've attempted to use on it.

\section*{
  Book Answer
}

In contrast with many other answers in the back of this book, the authors dedicated two pages to this answer, but really only one as they spend a page and a half describing the links the problem has to a more general problem called the ballot problem.  The answer on the first page describes but makes no attempt to justify called the "reflection principle" here, or the reflection lemma in other places I have found while researching the meaning of this paragraph.

\par

The book describes a method for taking an inadmissible sequence and finding the unique 'reflected' admissible sequence of length one more than the inadmissible one corresponding to it.

\end{document}