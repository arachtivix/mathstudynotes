\documentclass{article}
\usepackage{amsmath}

\title{The Art of Computer Programming -- Chapter 2.2, problem 2}
\author{Daniel Werner}

\begin{document}

\maketitle

\section*{
    Problem Statement
}

Is it possible to use a stack to permute the input sequence 1 2 3 4 5 6 into the output sequence 3 2 5 6 4 1?  Into 1 5 4 6 2 3?

\section*{
    Analysis
}

The sequence 3 2 5 6 4 1 seems doable.  Treating the state of the scenario between manipulations as three partitioned sequence, the middle being the stack with its top to the left side:
\newline
| | 1 2 3 4 5 6
\newline
| 1 | 2 3 4 5 6
\newline
| 2 1 | 3 4 5 6
\newline
| 3 2 1 | 4 5 6
\newline
3 | 2 1 | 4 5 6
\newline
3 2 | 1 | 4 5 6
\newline
3 2 | 4 1 | 5 6
\newline
3 2 | 5 4 1 | 6
\newline
3 2 5 | 4 1 | 6
\newline
3 2 5 | 6 4 1 |
\newline
3 2 5 6 | 4 1 |
\newline
3 2 5 6 4 | 1 |
\newline
3 2 5 6 4 1 | |
\newline

Trying to get 1 5 4 6 2 3, there are troubles:
\newline
| | 1 2 3 4 5 6
\newline
| 1 | 2 3 4 5 6
\newline
1 | | 2 3 4 5 6
\newline
1 | 2 | 3 4 5 6
\newline
1 | 3 2 | 4 5 6
\newline
1 | 4 3 2 | 5 6
\newline
1 | 5 4 3 2 | 6
\newline
1 5 | 4 3 2 | 6
\newline
1 5 4 | 3 2 | 6
\newline
1 5 4 | 6 3 2 |
\newline
1 5 4 6 | 3 2 |

The only way to get the 2 into the output before the 3 would be if 3 stayed in the stack and 2 bypassed it, but both needed to be in the stack because they both emerge before 4, which is already in the output.  It would seem if 3 and 2 are to be inverted in a doable output sequence, it would need to happen earlier in the sequence than this.

\par

I've also written some python code, included in this folder, to simulate this process. Just as a matter of implementation, it seems a little tricky just to keep the order of the output right, as the problem asks you to consider the order on the track and not the order of output, which are reverses of one another as the problem is framed.  I did much the same kind of thing the first time around when attempting the manual scenario above.  The script outputs all valid permutations for the scenario of 1 2 3 4 5 6, but can easily be parameterized with almost any other input sequence, depending on your patience and computational resources.

\end{document}
