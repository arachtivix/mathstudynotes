\documentclass{article}
\usepackage{amsmath}

\title{AOCP v1 ch 1.3.1 Problem 16}
\author{Daniel Werner}

\begin{document}

\maketitle

\section*{
    Problem Statement
}

Write a mix program that writes memory cells 0000-0099 to zero.

\section*{
    Analysis
}

For a verifiable, practical implementation, this program
 will have to load something into the memory cells that 
 can be read to check, rather than zeros, but the idea is
 the same.  For the purposes of this exercise I'll start
 with some smaller range of memory locations as the size
 of the data is hardly the point here.  This problem will 
 be a good way to get some of the fundamentals of the GNU emulator down.

\par

My first successful attempt at a solution to this problem
basically ignored the author's suggestion to use MOVE.
It uses the first numbered register as a counter and stores
the empty value of the A register into memory at the
position stored in the counter ('CTR') variable.  I could
probably have used indexing by the numbered register rather
than the variable.  I've also taken one of the example
scripts from the GNU page and printed some of the relevant
parts of the memory as part of a write-run-evaluate workflow
that ended up working for me quite nicely once everything
worked.

\par

The second solution uses MOVE and is quite a bit simpler.
The book mentions something that is going over my head a bit
on assuming byte sizes, but setting the first numbered
register to the start of an empty range of memory of the
same size as the target memory range and subsequently
invoking MOVE on the desired length seems to do fine in the
emulator.

\end{document}