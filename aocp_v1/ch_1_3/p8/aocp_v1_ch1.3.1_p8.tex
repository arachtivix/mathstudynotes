\documentclass{article}
\usepackage{amsmath}
\usepackage{nicematrix}

\title{AOCP V1 Ch 1.3.1 P8}
\author{Daniel Werner}

\begin{document}

\maketitle

\section*{Problem Statement}

The text refers to the table on page 133, which contains
the following as an example of what might happen when
a mixal line 'DIV 1000' is applied under the given
circumstances:

\vspace*{1em}

\begin{tabular}{|c|c|c|c|c|c|c|}
    \hline
    + & \multicolumn{5}{c|}{0} & rA before \\
    \hline
    $+$ & \multicolumn{2}{c|}{1235} & 0 & 3 & 1 & rX before \\
    \hline
    $-$ & 0 & 0 & 0 & 2 & 0 & Cell 1000 \\
    \hline
    $+$ & 0 & \multicolumn{2}{c|}{617} & ? & ? & rA after \\
    \hline
    $-$ & 0 & 0 & 0 & ? & 1 & rX after \\
    \hline
\end{tabular}

\vspace*{1em}

The question asks what if rX before instead contained

\vspace*{1em}

\begin{tabular}{|c|c|c|c|c|c|c|}
    \hline
    $-$ & \multicolumn{2}{c|}{1234} & 0 & 3 & 1 & rX before \\
    \hline
\end{tabular}

\section*{Analysis}

The combined rA and rX registers represent the quantity to
be divided.  The book says the sign for the rX register is
ignored, so the significance of the sign in the proposed
change to the rX register should not change that part of the
operation.

\par

The odd behavior of the original operation, where the combined
1 and 2 fields seem to cause a result in the rA after, seem
to point to a detail that the author wants the reader to be
aware of.  The operation unterdaken, 'DIV 1000' does not
specify fields, which means it should be applied to all fields.
Presumably when an operation such as DIV is applied to all
fields, the grouping of fields in the representation on paper
does not affect the numerical operation.  Yet there are question
marks on the rA after, which indicate that the operation will
be represented differently depending on the actual underlying
base of implementation for a given mix computer.

In order to clarify the behavior of this operation, the
same scenario can be written explicitly in binary and decimal
forms for comparison.

\subsection*{The Binary Interpretation}

\begin{figure}
    \begin{center}
        \begin{tabular}{|c|c|c|c|c|c|c|}
            \hline
            + & 000000 & 000000 & 000000 & 000000 & 000000 & rA Before \\
            \hline
            $+$ & 010011 & 010011 & 000000 & 000011 & 000001 & rX before \\
            \hline
            $-$ & 000000 & 000000 & 000000 & 000010 & 000000 & Cell 1000 \\
            \hline
            $+$ & 000000 & 001001 & 101001 & ? & ?? & rA after \\
            \hline
            $-$ & 000000 & 000000 & 000000 & ??? & 000001 & rX after \\
            \hline
        \end{tabular}
    \end{center}
    \caption{binary interpretation of the original operation}
    \label{fig:binquestion}
\end{figure}

Figure \ref{fig:binquestion} represents the same data as the
original table, only having the quantities fully spelled out
in binary format.  For example 1235 has been converted to
10011010011, or two successive binary
mix computer 'bytes' of 010011 then 010011.  Similarly, interpreting 617
as binary yields 1001101001, or mix bytes of 001001 then 101001. The
'?' entries have been retained in the figure, but with the base
fixed they do come out to specific values.

\begin{align*}
    rAX_{before} = &+000000,000000,000000,000000,000000,010011,\\
    &010011,000000,000011,000001_2 \\
    cell1000 = &000000,000000,000000,000010,000000_2 \\
    rA_{after} = &+000000,001001,101001,100000,000001_2 \\
    rX_{after} = &000000,000000,000000,000001,000001_2
\end{align*}

So for a base 2 mix machine, as per the placeholders in
figure \ref{fig:binquestion}, $? = 100000_2 = 32_{10}$, $?? = 000001$,
$??? = 000001$.  Looking at $rA_{after}$:

\begin{equation*}
    rA_{after} (2:3) = 001001,101001_2 = 617_{10}
\end{equation*}

The apparently strange consequence of the DIV operation
becomes clear: it's just doing the integer division operation
on the whole memory location and the authors have chosen to
present the results grouping (2:3) together.

\subsection*{The Decimal Interpretation}

The decimal interpretation is more natural, but still requires
some notation to signify the role that the mix 'byte' plays
in figure \ref{fig:decquestion}.

\begin{figure}
    \begin{center}
        \begin{tabular}{|c|c|c|c|c|c|c|}
            \hline
            + & 00 & 00 & 00 & 00 & 00 & rA before \\
            \hline
            $+$ & 12 & 35 & 00 & 03 & 01 & rX before \\
            \hline
            $-$ & 00 & 00 & 00 & 02 & 00 & Cell 1000 \\
            \hline
            $+$ & 00 & 06 & 17 & ? & ?? & rA after \\
            \hline
            $-$ & 00 & 00 & 00 & ??? & 01 & rX after \\
            \hline
        \end{tabular}
    \end{center}
    \caption{decimal interpretation of the original operation}
    \label{fig:decquestion}
\end{figure}

\begin{align*}
    rAX_{before} = &+00,00,00,00,00,12,35,00,03,01_{10} \\
    = &1235000301_{10} \\
    cell1000 = &00,00,00,02,00_{10} \\
    = &200_{10} \\
    rA_{after} = &| \lfloor 1235000301_{10} / 200_{10} \rfloor | = 6175001_{10} \\
    = & 00,06,17,50,01_{10} \\
    rX_{after} = &| 1235000301_{10} mod 200_{10} | \\
    = & 01,01_{10}
\end{align*}

For the decimal interpretation, $? = 50$, $?? = 01$, $??? = 01$.

\subsection*{Answering the Question}

The modified value for rX is:

\vspace*{1em}

\begin{tabular}{|c|c|c|c|c|c|c|}
    \hline
    $-$ & \multicolumn{2}{c|}{1234} & 0 & 3 & 1 & rX before \\
    \hline
\end{tabular}

\vspace*{1em}

which corresponds with the base-10 encoded memory value of
-12,34,00,03,01.  Using the decimal interpretation:

\begin{align*}
    rAX_{before} = &00,00,00,00,00,12,34,00,03,01_{10} \\
    cell1000 = &00,00,00,02,00_{10} \\
    rA_{after} = &00,06,17,00,01_{10} \\
    rX_{after} = &00,00,00,01,01_{10} \\
\end{align*}

So the value of $rAX_{after}$(2:3) actually did not change with the
new value of $rX_{before}$. The lower parts of the output are
changed.


\end{document}