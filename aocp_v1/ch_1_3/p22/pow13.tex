\documentclass{article}
\usepackage{amsmath}

\title{AOCP v1 ch 1.3.1 Problem 22}
\author{Daniel Werner}

\begin{document}

\maketitle

\section*{
    Problem Statement
}

Write a program that computes $x^{13}$ using the fewest instructions.  Write a
program that computes $x^{13}$ using the least running time.  The value of $x$
can be found in memory cell 2000.  The result should be placed in register A
when the program execution ends.

\section {
    Analysis
}

The maximum possible value for a mix register or a memory cell
is 1073741824 and because this system uses an explicit sign,
the minimum is -1073741824.  The limited numerical capacity ensures
that there will be very few integer input values that will have
results that will fit into the mix register.  For positive $x$, the largest
possible value whose thirteenth power will fit in such a cell
or register will be $4^{13}=67108864$ because $5^{13}=1220703125 >
1073741824$.

\subsection {
  Lookup Table Approach
}
The limited set of valid inputs means the problem is susceptible to a
simple lookup table approach, which should suffice for the
execution time version in any case and possibly for the
program size one if we cannot get another version to less
than the size of the lookup table plus a couple instructions
to handle the lookup.  In my estimation, this puts a ceiling
for the instruction size version at about 11 or 12 instructions
as the lookup table will be 9 and all we really need is to
set an offset and copy to the A register based on the offset.

\par

After implementation, the run time seems to have come to 20 $u$ given my
first cut implementation and 15 program instructions including the
lookup table.  I spend a lot more time developing the
test harness than I did with the mixal program because I would like
to use it for automating the analysis.  The lookup approach can be
found in pow13\_lookup.scm in this folder.

\subsection {
  Iteration Approach
}

The second approach I took was to load the value into register A then
to run MUL against register A 13 times, shifting the results back into
the A register each time.  This approach yielded a program that runs in
156 $u$ and has an instruction count of 27. So far the lookup table is
doing better on both metrics.

\subsection {
  Loop Approach
}

The third approach I took was to load the value into register A and build
a loop that would call MUL then check a counter and loop back until twelve
runs of the loop had taken place.  The execution time was 205, which means
the looping instructions added 49 $u$ for a given run.  This program only
took 10 lines to write, so it takes the shortest program category.

\section {
  Book Solutions
}

The solutions in the back of the book definitely contained more useful
information.  The book spends nearly a page in the back on explanations
and I'm taking that as some measure of the worthiness of trying to
understand it, since other answers often seem almost dismissively terse.

\subsection {
  Minimum Time Solution Without Lookup Table
}

The book offered two minimum time solutions: a program that used different
parts of the same register to calculate sub-parts of the final result in parallel
and a separate program that used a lookup table.  The solution not involving
a lookup table uses some clever tricks to get MUL calls to do more than one
calculation effectivley upon different subsets of the fields at one time.
This solution illustrates how shift operators can be used to facilitate this
kind of optimization.

\subsection {
  Minimum Instructions
}

\end{document}
