\documentclass{article}
\usepackage{amsmath}

\title{AOCP v1 ch 1.3.1 Problem 22}
\author{Daniel Werner}

\begin{document}

\maketitle

\section*{
    Problem Statement
}

Write a program that computes $x^{13}$ using the fewest instructions.  Write a
program that computes $x^{13}$ using the least running time.  The value of $x$
can be found in memory cell 2000.  The result should be placed in register A
when the program execution ends.

\section*{
    Analysis
}

The maximum possible value for a mix register or a memory cell
is 1073741824 and because this system uses an explicit sign,
the minimum is -1073741824.  The limited numerical capacity ensures
that there will be very few integer input values that will have
results that will fit into the mix register.  For positive $x$, the largest
possible value whose thirteenth power will fit in such a cell
or register will be $4^{13}=67108864$ because $5^{13}=1220703125 >
1073741824$.

\subsuection* {
  Lookup Table Approach
}
The limited set of valid inputs means the problem is susceptible to a
simple lookup table approach, which should suffice for the
execution time version in any case and possibly for the
program size one if we cannot get another version to less
than the size of the lookup table plus a couple instructions
to handle the lookup.  In my estimation, this puts a ceiling
for the instruction size version at about 11 or 12 instructions
as the lookup table will be 9 and all we really need is to
set an offset and copy to the A register based on the offset.

\par

After implementation, the run time seems to have come to 20 given my
first cut implementation and 15 program instructions including the
lookup table.  I spend a lot more time developing the
test harness than I did with the mixal program because I would like
to use it for automating the analysis.

\subsection* {
  Iteration Approach
}

forthcoming

\end{document}
