\documentclass{article}
\usepackage{amsmath}

\title{AOCP v1 ch 1.3.1 Problem 21}
\author{Daniel Werner}

\begin{document}

\maketitle

\section*{
    Problem Statement
}

(a) Can the J register ever be zero? (b) Write a program that given
a number $N$ in rI4 sets register J to $N$, assiming
$0 < N \le 3000$.  Your program should start at location 3000.
When your program stops its execution, the contents of the memory
cells must be unchanged.

\section*{
    Analysis
}

Part (a) indicates that there is something special about the J register
that might stop it from being equal to zero.  During normal execution not
after a jump instruction, this seems like it should be true, as the
lowest position a program could be executing from would be position 0,
with the jump register being 1.  However, considering jump instructions,
why would it not be possible to have a jump instruction set the J register
to zero, indicating the next instruction after the jump is in memory cell
zero?

\par

Part (b) seems like it should be well suited to answer the question in part
(a), but it is phrased in a way that suggests an unexpected answer, given the
paragraph above, or at least seems to leave room for either way you migh
answer to be true.

\par

The phrasing of part (b) suggests something external to the program setting
up the value in rI4, which seems to call for some light scripting.

\end{document}