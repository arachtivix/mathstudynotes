\documentclass{article}
\usepackage{amsmath}

\title{AOCP v1 ch 1.3.1 Problem 21}
\author{Daniel Werner}

\begin{document}

\maketitle

\section*{
    Problem Statement
}

(a) Can the J register ever be zero? (b) Write a program that given
a number $N$ in rI4 sets register J to $N$, assiming
$0 < N \le 3000$.  Your program should start at location 3000.
When your program stops its execution, the contents of the memory
cells must be unchanged.

\section*{
    Analysis
}

Part (a) indicates that there is something special about the J register
that might stop it from being equal to zero.  During normal execution not
after a jump instruction, this seems like it should be true, as the
lowest position a program could be executing from would be position 0,
with the jump register being 1.  However, considering jump instructions,
why would it not be possible to have a jump instruction set the J register
to zero, indicating the next instruction after the jump is in memory cell
zero?

\par

Part (b) seems like it should be well suited to answer the question in part
(a), but it is phrased in a way that suggests an unexpected answer, given the
paragraph above, or at least seems to leave room for either way you might
answer to be true.

\par

The phrasing of part (b) suggests something external to the program setting
up the value in rI4, which seems to call for some light scripting. The script
in this folder loads the memory into a Sheme list, loads the program into the
mix vm, sets the I4 register to some in-range value, runs the program, then
compares the memory to the Scheme list and the value in the jump register to
the value placed in the I4 register.  The output either confirms or rejects
the outcome.

\subsection* {
  Book Solution
}

I had to look up the answer to this question in the back of the book because
I could not figure out how to get a program to stop if it needed to terminate
on an arbitrary memory location. My initial ideas for the solution revolved
around keeping any small changes that needed to occur in memory cells in
registers as temporary values to be placed once the execution was drawing
to a close, but the stopping question was where I really got stuck. Unintentionally
I also found that programs that do not stop cause Emacs to get stuck as well,
but perhaps I am just too new to Emacs to know what to do when this happens.

\par

The book solution does not contain any HLT instruction, so the means of stopping
the execution is not obvious.  The script I wrote nonetheless confirms that it
does in fact end and that it fulfills the other requirements as well.  After stepping through
this program on the emulator it becomes much clearer how weird this program is:
it has six instructions, each executing exactly once, and yet it has a run time 
of over a thousand.  Unset memory cells have a value of zero, which evaluates to
NOP, which the vm happily follows from the end of the program up to the last
memory cell.  When the program reaches the last memory cell, it ends.  It's also
worth pointing out that the jump register stays set up to the end, despite going
one by one through all the imputed NOPs.

\par

It seems perhaps the problem writer may have wanted the reader to know about this 
unusual method of stopping the program.  It's probably specified in the book and
I just forgot it or something, so it's a good thing to know from experience gained
by doing this problem I guess.  The last observation I guess I'll make here is
that the MIX instruction written to the memory cell denoted by the R4 value is set
to a JMP instruction to the last instruction in the program.  That last instruction 
then fills that same designated memory cell again with its previous value, allowing
the virtual machine to dance up the NOPs and complete.

\par

There were certainly parts of this problem that taught me a new things, but I am glad I
did not spend much time trying to solve it because of the interesting interpretation
involved in calling this a program that actually terminates.  My instincts I feel
were more or less validated because I would not have imagined the scenario envisioned
by this solution as being a clear way of meeting the acceptance criteria.  I would say
maybe this carries if we can assume the memory range of 3007 to 3999 is in fact filled
with zeros, but this seems like the very kind of thing the book writers might not want
you to assume, so it is not particularly convincing to me yet.

\end{document}
