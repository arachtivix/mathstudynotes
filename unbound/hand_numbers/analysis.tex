\documentclass{article}
\usepackage{amsmath}

\title{Ideal Numerical Mapping for Unordered Card Selections}
\author{Daniel Werner}

\begin{document}

\maketitle

The question here is what method can be used to map a selection of $n$ playing cards from the standard deck of 52?  Starting out we can say that a selection will be a sequence of numbers where each suit and rank is one of the numbers from 0 to 51. The mapping of number to suit and rank is irrelevant and could be any arbitrary one-to-one correspondence.  The order of the selection is not important, so there are ${52 \choose n}$ ways to get a selection of $n$ cards as long as there are no other limitations.  This then also means we can treat the sequence as a monotonically decreasing sequence $\{\alpha,\beta,...\omega\}$ where $\alpha > \beta > ... > \omega$.

\par

\section*{Hand to Number}

Given a hand and finding its numerical value, another way to ask this same question might be asking that given a means of comparing selections where different selections have different values, how many selections are less than the one given. For a one card hand, this is easy:

\begin{equation}
    f(\{n\}) = n - 1
\end{equation}

Here we just use the numerical value of the card directly as the hand ranking.  For the purposes of the sort, let's say the maximum hand for a deck size $d$ and a selection of $k$ cards is the sequence of numbers:

\begin{equation}
    f(\{d - 1, d - 2, ..., d - k\}) = {d \choose k} - 1
\end{equation}

where the sequence above is contiguous from $d - 1$ down to $d - k$ inclusive.  This is because the top hand will be the only hand of the set of possibilities that is not lower.

\par

For two card hands, there will be a higher card and a lower card.  With the higher card at $\alpha$ and a lower card $\beta$, there are $\beta$ ways to make lesser hands with $\alpha$ still being the higher card.  Then there are all the hands one can make with two cards with both cards less than $\alpha$, comprised of $\{\alpha - 1, \alpha -2\}$ and $f(\{\alpha - 1, \alpha -2\})$, the latter of which is of course defined as the number of hands less than $\{\alpha - 1, \alpha -2\}$. With some simplification:

\begin{equation}
    f(\{\alpha, \beta\}, d) = \beta + 1 + f(\{\alpha - 1,\alpha - 2\})
    = \beta + {\alpha \choose 2}
\end{equation}

Trying some small values, there are three two-card hands from a deck of three: $\{2,1\},\{2,0\}, \{1,0\}$ in that order, so we would expect those pairs to resolve to 2, 1, and 0 respectively.

\begin{align*}
    f(\{2,1\}, 3) &= 1 + {2 \choose 2} = 2 \\
    f(\{2,0\}, 3) &= 0 + {2 \choose 2} = 1 \\
    f(\{1,0\}, 3) &= 0 + {1 \choose 2} = 0
\end{align*}

Trying out hands of 2 from a deck of 4: $\{3,2\},\{3,1\},\{3,0\},\{2,1\},\{2,0\},\{1,0\}$ is the order so:

\begin{align*}
    f(\{3,2\}) = 2 + {3 \choose 2} = 5 \\
    f(\{3,1\}) = 1 + {3 \choose 2} = 4 \\
    f(\{3,0\}) = 0 + {3 \choose 2} = 3 \\
    f(\{2,1\}) = 1 + {2 \choose 2} = 2 \\
    f(\{2,0\}) = 0 + {2 \choose 2} = 1 \\
    f(\{1,0\}) = 0 + {1 \choose 2} = 0 \\
\end{align*}

For three card hands we can use the same kind of reasoning again.

\begin{equation}
    f(\{\alpha, \beta, \gamma\}, d) =
\end{equation}

\end{document}