\documentclass{article}
\usepackage{amsmath}
\usepackage{amsfonts}
\usepackage{hyperref}

\title{LibreTexts Math Partial Fractions}
\author{Daniel Werner}

\begin{document}

\maketitle

\section*{Exercise 11.4.1 Problem Statement}

Find the partial fraction decomposition of the following expression.

\begin{equation*}
    \frac{x}{(x-3)(x-2)}
\end{equation*}

\section*{Solution}


\begin{equation*}
    \frac{x}{(x-3)(x-2)} = \frac{A}{x-3} + \frac{B}{x-2}
\end{equation*}
\begin{equation*}
    x = A \cdot (x-2) + B \cdot (x - 3)
\end{equation*}
\begin{equation*}
    x = Ax - 2A + Bx - 3B
\end{equation*}
\begin{equation*}
    x = (A + B)x - 2A - 3B
\end{equation*}
\begin{equation*}
    A + B = 1
\end{equation*}
\begin{equation*}
    2A + 3B = 0
\end{equation*}
Subtracting one equation from the other:
\begin{equation*}
    2A + 3B = 0
\end{equation*}
\begin{equation*}
    2A + 2B = 2
\end{equation*}
\begin{equation*}
    B = -2
\end{equation*}
\begin{equation*}
    A = 3
\end{equation*}
\begin{equation*}
    \frac{x}{(x-3)(x-2)} = \frac{3}{x-3} - \frac{2}{x-2}
\end{equation*}

\section*{Exercise 11.4.2 Problem Statement}

Find the partial fraction decomposition of the expression with repeated linear factors.

\begin{equation*}
    \frac{6x - 11}{(x-1)^2}
\end{equation*}

\section*{Solution}

\begin{equation*}
    \frac{6x - 11}{(x-1)^2} = \frac{A}{x - 1} + \frac{B}{(x - 1)^2}
\end{equation*}
\begin{equation*}
    6x - 11 = (x - 1) \cdot A + B
\end{equation*}
\begin{equation*}
    6x - 11 = x A + (B - A)
\end{equation*}
\begin{equation*}
    A = 6
\end{equation*}
\begin{equation*}
    B - A = -11
\end{equation*}
\begin{equation*}
    B = -5
\end{equation*}
\begin{equation*}
    \frac{6x - 11}{(x-1)^2} = \frac{6}{x - 1} - \frac{5}{(x - 1)^2}
\end{equation*}

\section*{Exercise 11.4.3 Problem Statement}

Find a partial fraction decomposition of the given expression.

\begin{equation*}
    \frac{5x^2 - 6x + 7}{(x-1)(x^2 + 1)}
\end{equation*}

\section*{Solution}

\begin{equation*}
    \frac{5x^2 - 6x + 7}{(x-1)(x^2 + 1)}
    =
    \frac{A}{x - 1} + \frac{Bx + C}{x^2 + 1}
\end{equation*}
\begin{equation*}
    5x^2 - 6x + 7
    =
    A (x^2 + 1) + (Bx + C)(x - 1)
\end{equation*}
\begin{equation*}
    5x^2 - 6x + 7
    =
    Ax^2 + A + Bx^2 - Bx + Cx - C
\end{equation*}
\begin{equation*}
    5x^2 - 6x + 7
    =
    (A + B)x^2 + (C - B)x + (A - C)
\end{equation*}
\begin{align*}
    &A + B = 5 \\
    &C - B = -6 \\
    &A - C = 7 \\ \\
    &A + B + C - B = 5 + (-6) \\
    &A + C = -1 \\
    &A + C + A - C = 7 + (-1) \\
    &2A = 6 \\
    &A = 3 \\
    &B = 2 \\
    &C = -4 \\
\end{align*}
\begin{equation*}
    \frac{5x^2 - 6x + 7}{(x-1)(x^2 + 1)}
    =
    \frac{3}{x - 1} + \frac{2x - 4}{x^2 + 1}
\end{equation*}

\end{document}