\documentclass{article}
\usepackage{amsmath}
\usepackage{nicematrix}

\title{Project Euler P6}
\author{Daniel Werner}

\begin{document}

\maketitle

\section*{Problem Statement}

https://projecteuler.net/problem=6

\section*{Solution}

\begin{equation*}
    \left( \sum_{0 < i \le n} i \right)^2 
    - \sum_{0 < i \le n} i^2 =
    \sum_{0 < i, j \le n} i j
    - \sum_{0 < i = j \le n} i^2
\end{equation*}

The above on the left represents the exact formulation
from the question.  The right represents a two-index
representation of the same.  The part with
$\sum_{0 < i \le n} i^2 = \sum_{0 < i = j \le n} i^2$
is not terribly far a journey.  The part with 
$\left( \sum_{0 < i \le n} i \right)^2
= \sum_{0 < i, j \le n} i j$ is the case simply because
the latter represents a sum with any two-tuple $i$ and
$j$ from the same range 1 to n because that is what
happens when you multiply two sums of terms: each
individual term forms a square in a grid and the sum of the
entire grid represents the given two-index expression.  Here
are some small values for later comparison:

\begin{align*}
    f_0 &= 0 \\
    f_1 &= 1^2 - 1^2 = 0 \\
    f_2 &= (1 + 2)^2 - (1^2 + 2^2) = 9 - 5 = 4 \\
    f_3 &= (1 + 2 + 3)^2 - (1^2 + 2^2 + 3^2) = 36 - 14 = 22 \\
    f_4 &= (1 + 2 + 3 + 4)^2 - (1^2 + 2^2 + 3^2 + 4^2) = 100 - 30 = 70
\end{align*}

A little more
manipulation of the sum demonstrates how you can recombine
parts of these expressions:

\begin{align*}
    \sum_{0 < i, j \le n} i j
    &=
    \sum_{0 < i=j \le n} i j
    + 2 \cdot \sum_{0 < i < j \le n} i j
    \\
    &=
    \sum_{0 < i = j \le n} i^2
    + 2 \cdot \sum_{0 < i < j \le n} i j
\end{align*}

Substituting in $\left( \sum_{0 < i, j \le n} i j \right)$
in the larger expression:

\begin{align*}
    \left( \sum_{0 < i \le n} i \right)^2 
    - \sum_{0 < i \le n} i^2 &=
    \sum_{0 < i = j \le n} i^2
    + 2 \sum_{0 < i < j \le n} i j
    - \sum_{0 < i = j \le n} i^2
    \\
    &= 2 \cdot \sum_{0 < i < j \le n} i j
\end{align*}

This manipulation actually turns out to be rather intuitive: the square of
the sum of the first n integers contains the sum of the first n squares,
then on top all of the other ordered two-tuples of 1 through n.  Since
there is an ij for every ji, we can just take each order-insensitive
two-tuple twice.  This sum in itself is quite simple to implement in code,
but we can go further.  Let's start by verifying some small values first though:

\begin{align*}
    f_n = 2 \cdot \sum_{0 < i < j \le n} i j
\end{align*}

\begin{align*}
    f_0 &= 2 \cdot 0 = 0 \\
    f_1 &= 2 \cdot 0 = 0 \\
    f_2 &= 2 \cdot (1 \cdot 2) = 4 \\
    f_3 &= 2 \cdot (1 \cdot 2 + 1 \cdot 3 + 2 \cdot 3) = 22 \\
    f_4 &= 2 \cdot (1 \cdot 2 + 1 \cdot 3 + 1 \cdot 4 \\
    & + 2 \cdot 3 + 2 \cdot 4 + 3 \cdot 4) = 70
\end{align*}

The small values check out nicely.  Considering the pattern setup in the checks
above shows that from step to step starting from $f_2$ the terms in the sum
remain the same, only adding new values from the last iteration, not changing
existing ones.  In $f_2$ we added $1 \cdot 2$.  In $f_3$ we added
$1 \cdot 3$ and $2 \cdot 3$.  In $f_4$ we added $1 \cdot 4$, $2 \cdot 4$,
and $3 \cdot 4$.

\end{document}