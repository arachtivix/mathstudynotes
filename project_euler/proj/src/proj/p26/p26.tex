\documentclass{article}
\usepackage{amsmath}
\usepackage{amssymb}
\usepackage{graphicx}

\title{Project Euler - Problem 26}
\author{Solution}
\date{2025-05-10}

\begin{document}

\maketitle

\section*{Problem 26: Reciprocal Cycles}

A unit fraction contains \\$1\\$ in the numerator. The decimal representation of the unit fractions with denominators \\$2\\$ to \\$10\\$ are given:\\
\\begin\\{align\\}\\
1/2 \\&= 0.5\\\\\\
1/3 \\&=0.(3)\\\\\\
1/4 \\&=0.25\\\\\\
1/5 \\&= 0.2\\\\\\
1/6 \\&= 0.1(6)\\\\\\
1/7 \\&= 0.(142857)\\\\\\
1/8 \\&= 0.125\\\\\\
1/9 \\&= 0.(1)\\\\\\
1/10 \\&= 0.1\\
\\end\\{align\\}\\
Where \\$0.1(6)\\$ means \\$0.166666\\cdots\\$, and has a \\$1\\$-digit recurring cycle. It can be seen that \\$1/7\\$ has a \\$6\\$-digit recurring cycle.\\
Find the value of \\$d \\lt 1000\\$ for which \\$1/d\\$ contains the longest recurring cycle in its decimal fraction part.

\section*{Solution}

% TODO: Document your solution approach here

\section*{Implementation}

% TODO: Explain key parts of your implementation

\end{document}
