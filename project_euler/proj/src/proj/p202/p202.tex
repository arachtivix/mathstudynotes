\documentclass{article}
\usepackage{amsmath}
\usepackage{amssymb}
\usepackage{listings}
\usepackage{xcolor}

% Define colors for syntax highlighting
\definecolor{stringcolor}{RGB}{206,145,120}
\definecolor{keywordcolor}{RGB}{197,134,192}
\definecolor{commentcolor}{RGB}{96,139,78}
\definecolor{functioncolor}{RGB}{220,220,170}
\definecolor{numbercolor}{RGB}{181,206,168}

% Configure Clojure language
\lstdefinelanguage{Clojure}{
  morekeywords={def,defn,let,if,do,map,reduce,filter,when,cond,case,loop,recur,fn,ns,require},
  morestring=[b]",
  morecomment=[l];,
  numbers=left,
  numberstyle=\tiny\color{gray},
  basicstyle=\ttfamily\small,
  breaklines=true,
  keywordstyle=\color{keywordcolor},
  stringstyle=\color{stringcolor},
  commentstyle=\color{commentcolor},
  identifierstyle=\color{functioncolor},
  numberstyle=\color{numbercolor},
  sensitive=false
}

% Set default language to Clojure
\lstset{language=Clojure}

% Other common packages
\usepackage{graphicx}
\usepackage{hyperref}
\usepackage{enumerate}
\usepackage{geometry}

% Page setup
\geometry{margin=1in}

\title{Project Euler - Problem 202}
\author{Solution}
\date{2025-06-23}

\begin{document}

\maketitle

\section*{Problem 202: Laserbeam}


Three mirrors are arranged in the shape of an equilateral triangle, with their reflective surfaces pointing inwards. There is an infinitesimal gap at each vertex of the triangle through which a laser beam may pass.

\par
Label the vertices $A$, $B$ and $C$. There are $2$ ways in which a laser beam may enter vertex $C$, bounce off $11$ surfaces, then exit through the same vertex: one way is shown below; the other is the reverse of that.

\par
\begin{center}
[Image: ]

\par
Note: Please refer to the original problem for this image at projecteuler.net
\end{center}
There are $80840$ ways in which a laser beam may enter vertex $C$, bounce off $1000001$ surfaces, then exit through the same vertex.

\par
In how many ways can a laser beam enter at vertex $C$, bounce off $12017639147$ surfaces, then exit through the same vertex?

\par


\section*{Solution}

% TODO: Document your solution approach here

\section*{Implementation}

% TODO: Explain key parts of your implementation

\end{document}
