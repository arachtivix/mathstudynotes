\documentclass{article}
\usepackage{amsmath}
\usepackage{amssymb}
\usepackage{graphicx}

\title{Project Euler - Problem 206}
\author{Solution}
\date{2024-02-21}

\begin{document}

\maketitle

\section*{Problem 206: Concealed Square}

Find the unique positive integer whose square has the form 1_2_3_4_5_6_7_8_9_0,
where each "_" is a single digit.

\section*{Solution}

The solution involves finding a number whose square matches a specific pattern. We can approach this by:

1. Defining the pattern we're looking for
2. Creating a generalized brute force function
3. Optimizing the search space

\section*{Implementation}

\subsection*{Generalized Brute Force Function}

The generalized brute force function is a key component that allows us to:
\begin{itemize}
\item Test the solution approach with smaller patterns
\item Reuse the code for similar pattern-matching problems
\item Verify correctness before scaling to the full problem
\end{itemize}

The implementation consists of several helper functions:

\begin{enumerate}
\item \texttt{max-dec-exp-from-pattern}: Converts a pattern to its maximum possible value by replacing wildcards with 9
\item \texttt{min-dec-exp-from-pattern}: Converts a pattern to its minimum possible value by replacing wildcards with 0
\item \texttt{dec-exp-to-bigint}: Converts a decimal expansion to a bigint
\item \texttt{patt-max} and \texttt{patt-min}: Convenience functions to get the maximum and minimum values for a pattern
\item \texttt{brute-force-generalized}: The main function that searches for numbers whose squares match the pattern
\end{enumerate}

The generalized brute force function works as follows:

\begin{verbatim}
(defn brute-force-generalized [p]
  (filter #(matches-pattern? p (dec-exp-int (* %1 %1))) 
         (range (bigint (math/sqrt (patt-min p))) 
                (+ 1 (bigint (math/sqrt (patt-max p)))))))
\end{verbatim}

This function:
\begin{itemize}
\item Takes a pattern as input (e.g., '(1 :_ 2))
\item Calculates the minimum and maximum possible values that could match the pattern
\item Takes the square root of these values to determine the search range
\item Filters numbers whose squares match the pattern
\end{itemize}

The advantage of this generalized approach is that we can test it with simpler patterns first, such as:
\begin{itemize}
\item 1\_2 (single wildcard)
\item 1\_2\_3 (multiple wildcards)
\item Small subsets of the full pattern
\end{itemize}

This allows us to verify the correctness of our pattern matching and number generation before attempting the full problem.

\end{document}