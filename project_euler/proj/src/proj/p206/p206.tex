\documentclass{article}
\usepackage{amsmath}
\usepackage{amssymb}
\usepackage{graphicx}

\title{Project Euler - Problem 206}
\author{Solution}
\date{2025-05-23}

\begin{document}

\maketitle

\section*{Problem 206: Concealed Square}


Find the unique positive integer whose square has the form 1\_2\_3\_4\_5\_6\_7\_8\_9\_0, where each “\_” is a single digit.

\par


\section*{Solution}

\subsection{Can I brute force it?}

One solution to this problem might be to try numbers $n$ in some range and test for the pattern.  Just from a rough inital
approximation one could try integers $\lfloor\sqrt{10^{20}}\rfloor = 10^{10} < n < \lfloor\sqrt{2*10^{20}}\rfloor$.  
The obvious problem with this approach though is this requires something around 10 trillion numbers to test.\\

It seems as per the form that $n^2$ ends in 0, so $n$ must have a 2 and a 5 in its factors, which then means the last
 underscore has to be a 0.  So really we're looking for an $m^2$ of the form 1\_2\_3\_4\_5\_6\_7\_8\_9 that can have
 that part tacked on at the end to arrive at $n$.  This would leave the range needing testing at hundreds of billions,
 but this is not good enough -- enough computing power could be applied but that's not the point. \\

Interestingly, we know that 2 and 5 will not feature as factors in $m$ because neither will produce a 9 at the end. \\

\subsection{The meaning of the form: or, a tedious and probably useless decimal expansion}

\begin{align*}
    n^2 &= 1 \cdot 10^{18} + \alpha_0 \cdot 10^{17} + 2 \cdot 10^{16} + \alpha_1 \cdot 10^{15} 
        + 3 \cdot 10^{14} + \alpha_2 \cdot 10^{13} + 4 \cdot 10^{12} + \alpha_3 \cdot 10^{11} \\
        &+ 5 \cdot 10^{10} + \alpha_4 \cdot 10^{9} + 6 \cdot 10^{8} + \alpha_5 \cdot 10^{7} 
        + 7 \cdot 10^{6} + \alpha_6 \cdot 10^{5} + 8 \cdot 10^{4} + \alpha_7 \cdot 10^{3} \\
        &+ 9 \cdot 10^{2} + \alpha_8 \cdot 10^{1} + 0 \cdot 10^{0} 
\end{align*}

So the even powers of 10 belong to the known coeficients, and the odds belong to the unknowns.  One might group
them, but it's not obvious how this will help.

\section*{Implementation}

% TODO: Explain key parts of your implementation

\end{document}
