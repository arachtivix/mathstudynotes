\subsection{Solution Approaches}

The code presents two solution approaches:

\subsubsection{First Approach}
\begin{lstlisting}
(defn sol1 [] (take 1 (filter is-root-of-hidden-sq? (range min-n max-n))))
\end{lstlisting}

This is a brute-force approach that:
\begin{itemize}
    \item Defines the maximum and minimum possible values based on the pattern
    \item Takes their square roots to get the search range
    \item Filters numbers whose squares match the pattern
\end{itemize}

\subsubsection{Slightly Narrower Approach}
\begin{lstlisting}
(defn is-m-sq?
  [m]
  (matches-pattern?
   '(1 :_ 2 :_ 3 :_ 4 :_ 5 :_ 6 :_ 7 :_ 8 :_ 9)
   (dec-exp-int (* m m))))

(defn sol2 [] 
  (take 1 (filter is-m-sq? (range min-m max-m))))
\end{lstlisting}

This is another approach that:
\begin{itemize}
    \item Uses an intermediate step mentioned in the problem writeup
    \item Reduces the search space by looking at a shorter pattern
    \item Still uses brute force but over a smaller range
\end{itemize}

Both approaches are noted to be time-consuming, suggesting that further optimization might be needed for an efficient solution.


\subsubsection{Generalized Brute Force Function}

The generalized brute force function is a key component that allows us to:
\begin{itemize}
\item Test the solution approach with smaller patterns
\item Reuse the code for similar pattern-matching problems
\item Verify correctness before scaling to the full problem
\end{itemize}

The implementation consists of several helper functions:

\begin{enumerate}
\item \texttt{max-dec-exp-from-pattern}: Converts a pattern to its maximum possible value by replacing wildcards with 9
\item \texttt{min-dec-exp-from-pattern}: Converts a pattern to its minimum possible value by replacing wildcards with 0
\item \texttt{dec-exp-to-bigint}: Converts a decimal expansion to a bigint
\item \texttt{patt-max} and \texttt{patt-min}: Convenience functions to get the maximum and minimum values for a pattern
\item \texttt{brute-force-generalized}: The main function that searches for numbers whose squares match the pattern
\end{enumerate}

The generalized brute force function works as follows:

\begin{lstlisting}
(defn brute-force-generalized [p]
  (filter #(matches-pattern? p (dec-exp-int (* %1 %1))) 
         (range (bigint (math/sqrt (patt-min p))) 
                (+ 1 (bigint (math/sqrt (patt-max p)))))))
\end{lstlisting}

This function:
\begin{itemize}
\item Takes a pattern as input (e.g., '(1 :\_ 3))
\item Calculates the minimum and maximum possible values that could match the pattern
\item Takes the square root of these values to determine the search range
\item Filters numbers whose squares match the pattern
\end{itemize}

The advantage of this generalized approach is that we can test it with simpler patterns first, such as:
\begin{itemize}
\item 1\_2 (single wildcard)
\item 1\_2\_3 (multiple wildcards)
\item Small subsets of the full pattern
\end{itemize}

This allows us to verify the correctness of our pattern matching and number generation before attempting the full problem.
