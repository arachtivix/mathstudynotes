\section{Implementation}

The implementation solves Project Euler Problem 206 which deals with finding a number whose square follows a specific pattern: 1_2_3_4_5_6_7_8_9_0, where underscores represent single digits.

\subsection{Core Functions}

\subsubsection{Decimal Expansion}
First, we have a function to convert integers to their decimal expansion:

\begin{lstlisting}
(defn dec-exp-int "decimal expansion for an integer"
  ([n] (dec-exp-int n '()))
  ([n prv] (let [q (quot n 10) 
                 r (rem n 10)
                 cur (cons r prv)]
             (if (= q 0) cur (recur q cur)))))
\end{lstlisting}

This function recursively breaks down a number into its decimal digits, building a list from right to left. It uses:
\begin{itemize}
    \item \texttt{quot} to get the quotient when dividing by 10
    \item \texttt{rem} to get the remainder (last digit)
    \item \texttt{cons} to build the list of digits
\end{itemize}