\subsection{Another formula for $n^2$}

Taking the difference between squares of adjacent powers, it's fairly straight forward enough to convince
oneself that

\begin{align*}
     n^2 = \sum_{1 \leq i \leq n} (2i - 1)
\end{align*}

In this case the recursive version of the formula is a more useful way of looking at it:

\begin{align*}
     &p_n = p_{n - 1} + 2n - 1 \\
     &p_0 = 0
\end{align*}

Another way of describing the $2n - 1$ part is "the odd numbers."  So squares are just stacks of sequential odd numbers:
$1 = 1$; $4 = 1 + 3$; $9 = 1 + 3 + 5$; $16 = 1 + 3 + 5 + 7$ and so on.  Then there's the idea that the digits of the odd
numbers repeat at the end, so taking the last digit of 1, 3, 5, 7, 9, 11, 13, 15, 17, 19, 21 we get 1, 3, 5, 7, 9,
1, 3, 5, 7, 9, 1 (...).  Given that there are only so many inputs to a single digit process, it stands to reason that the
last digit of square numbers should end up in a repeating pattern.  For just the last digits this pattern goes:

\begin{center}
\begin{tabular}{ c c c c c }
$n$ & $n^2$ & $n^2 mod 10$ & $2n + 1$ mod 10 & next \\ 
1 & 1 & 1 & 3 & $(1 + 3) \mod 10 \rightarrow 4$ \\  
2 & 4 & 4 & 5 & $(4 + 5) \mod 10 \rightarrow 9$ \\
3 & 9 & 9 & 7 & $(9 + 7) \mod 10 \rightarrow 6$ \\
4 & 16 & 6 & 9 & $(6 + 9) \mod 10 \rightarrow 5$ \\
5 & 25 & 5 & 1 & $(5 + 1) \mod 10 \rightarrow 6$ \\
6 & 36 & 6 & 3 & $(6 + 3) \mod 10 \rightarrow 9$ \\
7 & 49 & 9 & 5 & $(9 + 5) \mod 10 \rightarrow 4$ \\
8 & 64 & 4 & 7 & $(4 + 7) \mod 10 \rightarrow 1$ \\
9 & 81 & 1 & 9 & $(1 + 9) \mod 10 \rightarrow 0$ \\
10 & 100 & 0 & 1 & $(0 + 1) \mod 10 \rightarrow 1$ \\
11 & 121 & 1 & 3 & $(1 + 3) \mod 10 \rightarrow 4$ \\
12 & 144 & 4 & 5 & $(4 + 5) \mod 10 \rightarrow 9$  
\end{tabular}
\end{center}

The table indicates that the last digit of the squares follows a simple, repeating recurence.  It repeats because
the limited input space must eventually result in the same state.  I'd express this recurrence as follows:

\begin{align*}
&D_1 = 1 \\
&D_n = (D_{n - 1} + (2n + 1)) \mod 10
\end{align*}

It seems to have a cycle length of 10, never hitting a value of 2, 3, 7, or 8, but traveling through the other digits.