\documentclass{article}
\usepackage{amsmath}
\usepackage{amssymb}
\usepackage{listings}
\usepackage{xcolor}

% Define colors for syntax highlighting
\definecolor{stringcolor}{RGB}{206,145,120}
\definecolor{keywordcolor}{RGB}{197,134,192}
\definecolor{commentcolor}{RGB}{96,139,78}
\definecolor{functioncolor}{RGB}{220,220,170}
\definecolor{numbercolor}{RGB}{181,206,168}

% Configure Clojure language
\lstdefinelanguage{Clojure}{
  morekeywords={def,defn,let,if,do,map,reduce,filter,when,cond,case,loop,recur,fn,ns,require},
  morestring=[b]",
  morecomment=[l];,
  numbers=left,
  numberstyle=\tiny\color{gray},
  basicstyle=\ttfamily\small,
  breaklines=true,
  keywordstyle=\color{keywordcolor},
  stringstyle=\color{stringcolor},
  commentstyle=\color{commentcolor},
  identifierstyle=\color{functioncolor},
  numberstyle=\color{numbercolor},
  sensitive=false
}

% Set default language to Clojure
\lstset{language=Clojure}

% Other common packages
\usepackage{graphicx}
\usepackage{hyperref}
\usepackage{enumerate}
\usepackage{geometry}

% Page setup
\geometry{margin=1in}

\title{Project Euler - Problem 203}
\author{Solution}
\date{2025-06-22}

\begin{document}

\maketitle

\section*{Problem 203: Squarefree Binomial Coefficients}


The binomial coefficients \\$\\displaystyle \\binom n k\\$ can be arranged in triangular form, Pascal's triangle, like this:

\par
\begin{center}
\begin{tabular}{|c|c|c|}
\hline
 & 1 &  \\
\hline
 & 1 &  & 1 &  \\
\hline
 & 1 &  & 2 &  & 1 &  \\
\hline
 & 1 &  & 3 &  & 3 &  & 1 &  \\
\hline
 & 1 &  & 4 &  & 6 &  & 4 &  & 1 &  \\
\hline
 & 1 &  & 5 &  & 10 &  & 10 &  & 5 &  & 1 &  \\
\hline
 & 1 &  & 6 &  & 15 &  & 20 &  & 15 &  & 6 &  & 1 &  \\
\hline
1 &  & 7 &  & 21 &  & 35 &  & 35 &  & 21 &  & 7 &  & 1 \\
\hline
\end{tabular}
\end{center}
.........

\par
It can be seen that the first eight rows of Pascal's triangle contain twelve distinct numbers: 1, 2, 3, 4, 5, 6, 7, 10, 15, 20, 21 and 35.

\par
A positive integer n is called squarefree if no square of a prime divides n.
Of the twelve distinct numbers in the first eight rows of Pascal's triangle, all except 4 and 20 are squarefree.
The sum of the distinct squarefree numbers in the first eight rows is 105.

\par
Find the sum of the distinct squarefree numbers in the first 51 rows of Pascal's triangle.

\par


\section*{Solution}

% TODO: Document your solution approach here

\section*{Implementation}

% TODO: Explain key parts of your implementation

\end{document}
