\documentclass{article}
\usepackage{amsmath}
\usepackage{nicematrix}

\title{Project Euler P6}
\author{Daniel Werner}

\begin{document}

\maketitle

\section*{Problem Statement}

See https://projecteuler.net/problem=361

\begin{align*}
    &T_0 = 0 \\
    &T_{2n} = T_n \\
    &T_{2n + 1} = 1 - T_{2n}
\end{align*}

wip

\section*{Solution}

For the purposes of the analysis, I will refer to the rules for T respectively by their occurrence as R1, R2, and R3.

\par

The Thue-Morse sequence is defined by a recurrence which does feature some predictable elements.  First, because it is not hard to find that $T_1 = 1$ (R1) and given that $T_{2n} = 1 - T_n$ (R3), it's not hard to accept that $T_{2^n} = 1$.  Next it looks like every value of $T$ can trace its path backwards to zero backwards using the rules from the problem statement, for example:

\begin{align*}
    T_{23} &= T_{\{10111\}_2} \\
    &= 1 - T_{11} \\ 
    &= 1 - (1 - T_5) \\
    &= 1 - (1 - (1 - T_2)) \\
    &= 1 - (1 - (1 - T_1)) \\
    &= 1 - (1 - (1 - (1 - T_0))) \\
    &= 1 - (1 - (1 - (1 - 0))) \\
    &= 0
\end{align*}

This example shows that the application of the rules go in the order R3, R3, R3, R2, R3, R1. Trying out another:

\begin{align*}
    T_{25} &= T_{\{11001\}_2} \\
    &= 1 - T_{12} \\ 
    &= 1 - T_6 \\
    &= 1 - T_3 \\
    &= 1 - (1 - T_1) \\
    &= 1 - (1 - (1 - T_0)) \\
    &= 1 - (1 - (1 - 0)) \\
    &= 1
\end{align*}

\par

In this case, the rule application goes R3, R2, R2, R3, R3, R1.  The relationship between the rule application and the binary expansion is clear: the first rule applied corresponds to the least significant digit in the binary expansion and it's R3 if that digit is 1, R2 if it is 0.  The rule applications proceed for each binary digit up in significance until the highest non-zero digit, then finish with R1 which does not correspond to any of the digits, or could be said to correspond with the zeros remaining in the binary expansion.  In this way it becomes clear that the determination of the value of $T_n$ depends entirely on whether an odd or an even number of R3 applications have taken place, odd meaning a resulting value of 1 and even meaning a resulting value of 0.

\par

wip

\end{document}