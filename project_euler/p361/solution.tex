\documentclass{article}
\usepackage{amsmath}
\usepackage{nicematrix}

\title{Project Euler P361}
\author{Daniel Werner}

\begin{document}

\maketitle

\section*{Problem Statement}

See https://projecteuler.net/problem=361

\begin{align*}
    &T_0 = 0 \\
    &T_{2n} = T_n \\
    &T_{2n + 1} = 1 - T_{2n}
\end{align*}

wip

\section*{Solution}

For the purposes of the analysis, I will refer to the rules for T respectively by their occurrence as R1, R2, and R3.

\par

The Thue-Morse sequence is defined by a recurrence which does feature some predictable elements.  First, because it is not hard to find that $T_1 = 1$ (R1) and given that $T_{2n} = 1 - T_n$ (R3), it's not hard to accept that $T_{2^n} = 1$.  Next it looks like every value of $T$ can trace its path backwards to zero backwards using the rules from the problem statement, for example:

\begin{align*}
    T_{23} &= T_{\{10111\}_2} \\
    &= 1 - T_{11} \\ 
    &= 1 - (1 - T_5) \\
    &= 1 - (1 - (1 - T_2)) \\
    &= 1 - (1 - (1 - T_1)) \\
    &= 1 - (1 - (1 - (1 - T_0))) \\
    &= 1 - (1 - (1 - (1 - 0))) \\
    &= 0
\end{align*}

This example shows that the application of the rules go in the order R3, R3, R3, R2, R3, R1. Trying out another:

\begin{align*}
    T_{25} &= T_{\{11001\}_2} \\
    &= 1 - T_{12} \\ 
    &= 1 - T_6 \\
    &= 1 - T_3 \\
    &= 1 - (1 - T_1) \\
    &= 1 - (1 - (1 - T_0)) \\
    &= 1 - (1 - (1 - 0)) \\
    &= 1
\end{align*}

\par

In this case, the rule application goes R3, R2, R2, R3, R3, R1.  The relationship between the rule application and the binary expansion is clear: the first rule applied corresponds to the least significant digit in the binary expansion and it's R3 if that digit is 1, R2 if it is 0.  The rule applications proceed for each binary digit up in significance until the highest non-zero digit, then finish with R1 which does not correspond to any of the digits, or could be said to correspond with the zeros remaining in the binary expansion.  In this way it becomes clear that the determination of the value of $T_n$ depends entirely on whether an odd or an even number of R3 applications have taken place, odd meaning a resulting value of 1 and even meaning a resulting value of 0.

\par

I think it would be helpful at this point to imagine a sequence F, such that $F_n$ is just a count of the number of 1s in the binary expansion of $n$ and combined with the findings from the previous paragraph, what this would mean for the value of $T_n$:

\begin{align*}
    F_0 = F_{\{0\}} &= 0 \rightarrow T_0 = 0 \\
    F_1 = F_{\{1\}} &= 1 \rightarrow T_1 = 1 \\
    F_2 = F_{\{10\}} &= 1 \rightarrow T_2 = 1 \\
    F_3 = F_{\{11\}} &= 2 \rightarrow T_3 = 0 \\
    F_4 = F_{\{100\}} &= 1 \rightarrow T_4 = 1 \\
    F_5 = F_{\{101\}} &= 2 \rightarrow T_5 = 0 \\
    F_6 = F_{\{110\}} &= 2 \rightarrow T_6 = 0 \\
    F_7 = F_{\{111\}} &= 3 \rightarrow T_7 = 1 \\
    F_8 = F_{\{1000\}} &= 1 \rightarrow T_8 = 1 \\
\end{align*}

Using the $F_n$ sequence in this way seems to be confirmed as it lines up perfectly with the examples for $T_n$ in the question.  Implementing $F_n$ in code made me realize that this new sequence is just $T_n$ except keeping an ongoing count.  Mathematically, one might express $F_n$ as:

\begin{align*}
    &F_0 = 0 \\
    &F_{2n} = F_n \\
    &F_{2n + 1} = F_n + 1
\end{align*}

This is easy to understand once you think of this as just two ways to build up a binary sequence: either put a 1 or a 0 into the least significant digit (rule 2 and 3 for $F_n$ respectively).  It's not amazingly clear to me how to handle this recurrence yet either, but maybe $\delta F_n$ will be useful.

\par

Let's say $\delta F_n = G_n$, which is the difference $F_n - F_{n - 1}$ and let's get $\delta\delta F_n = J_n$ as well just for kicks:
\begin{center}
    \begin{tabular}{c | c | c | c | c | c | c}
        n & $T_n$ & $F_n$ & $b_{n-1}$ & $b_{n}$ & $G_n$ & $J_n$ \\
        \hline
        0 & 0 & 0 & - & 0 & - & - \\
        1 & 1 & 1 & 0 & 1 & 1 & - \\
        2 & 1 & 1 & 01 & 10 & 0 & -1 \\
        3 & 0 & 2 & 10 & 11 & 1 & 1 \\
        4 & 1 & 1 & 011 & 100 & -1 & -2 \\
        5 & 0 & 2 & 100 & 101 & 1 & 2 \\
        6 & 0 & 2 & 101 & 110 & 0 & -1 \\
        7 & 1 & 3 & 110 & 111 & 1 & 1 \\
        8 & 1 & 1 & 0111 & 1000 & -2 & -3 \\
        9 & 0 & 2 & 1000 & 1001 & -1 & 1 \\
        10 & 0 & 2 & 1001 & 1010 & 0 & 1 \\
        11 & 1 & 3 & 1010 & 1011 & 1 & 1 \\
        12 & 0 & 2 & 1011 & 1100 & -1 & -2 \\
    \end{tabular}
\end{center}

These sequences seem rather unruly.  Not only do the differences make any discernable pattern, the differences also continue to fluctuate from negative to positive and back.  Perhaps attempting to reduce it to a simpler form is not going to happen.

\par

Another way I can see of looking at this is as frames between the powers of 2.  Since we know $T_{2^n} = 1$, that part is boring and the interesting parts lay between:

\begin{center}
    0 \\
    1 \\
    10 \\
    1001 \\
    10010110 \\
    1001011001101001 \\
    10010110011010010110100110010110 \\
\end{center}

This figure was easy to assemble because the whole structure is just a binary tree where each pair of entries is just the children of an entry on the layer above.  1 goes to 10 and 0 goes to 01.  It is from this view it becomes clear that you'll never see three 1's or three 0's in a row.  Each child pair is either 01 or 10 and no combination of 01 and 10 will ever contain any triple.  The whole sequence is composed of 01 and 10 parts put on as units.  Therefore no $n$ that contains a triple will ever be present in $T_n$.  It strikes me as necessary to demonstrate that no other limiting factor is present here, but for now it may be useful to see what results we get back for just removing binary expansions that contain three or more repeat bits.

\par

Although the sequence for $A$ is not fully constructed, it could be helpful now to just list some numbers that will appear in $A$ and some that will not:

\begin{center}
    \begin{tabular}{c | c | c}
        n & $b_{n}$ & appears in $A_n$ \\
        \hline
        0 & 0 & yes \\
        1 & 1 & yes \\
        2 & 10 & yes \\
        3 & 11 & yes \\
        4 & 100 & yes \\
        5 & 101 & yes \\
        6 & 110 & yes \\
        7 & 111 & no \\
        8 & 1000 & no \\
        9 & 1001 & yes \\
        10 & 1010 & yes \\
        11 & 1011 & yes \\
        12 & 1100 & yes \\
        13 & 1101 & yes \\
        14 & 1110 & no \\
        15 & 1111 & no \\
        16 & 10000 & no \\
        17 & 10001 & no \\
        18 & 10010 & yes \\
    \end{tabular}
\end{center}

A few ideas rolling around:
\begin{itemize}
    \item $2^n$ can never be in $A$ for any value $n > 2$ and neither can $2^n-1$ for any value of $n > 2$
    \item Any number whose remainder after dividing by $2^4$ that comes out as 7 will have the last three digits make the number impossible to get in $A$
    \item Any number not in the point above whose remainder after dividing by $2^5$ is greater than or equal to 14 will be impossible.
\end{itemize}

\par

wip

\end{document}